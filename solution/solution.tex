\documentclass{ctsol}

\title{ACM算法与微应用实验室2021年11月月赛题解}
\date{2021年12月1日}

\begin{document}
\maketitle
\addsolution{克隆干员}{AgOH}{模拟}
\addsolution{中转站}{AgOH\&Tifa}{前缀和}
\addsolution{三斜求积术}{AgOH}{模拟}
\addsolution{子树大小}{AgOH}{dfs}
\addsolution{雷立方体阵列}{Tifa}{线段树}
\addsolution{Go}{AgOH}{并查集}

\section*{题目概览}
\solutiontab

\section*{鸣谢}
感谢\href{https://github.com/Tiphereth-A}{\@Tifa}大佬参与本次比赛的出题工作。

\makesolution
\section*{做法}
模拟过程。比较简便的做法如下:

首先我们先将干员的四种不同朝向时的状况分别保存于数组中(也就是先把干员旋转好),然后开出一个初始值为0的数组用来表示战场,最后每输入一个点后就向这个战场数组中涂色(填1)即可。

\section*{标程}
\begin{lstlisting}
#include <iostream>
#include <string>
using namespace std;
string s[10];
int fw[5][10][10],a[15][15]; // fw[i] i从1到4分别保存干员朝向上下左右时的攻击范围;a为战场
int main()
{
    for(int i=0;i<7;i++) cin>>s[i]; // 用string简单读入
    for(int i=1;i<=7;i++) for(int j=1;j<=7;j++) fw[1][i][j]=s[i-1][j-1]-'0'; // 朝上
    for(int i=1;i<=7;i++) for(int j=1;j<=7;j++) fw[2][i][j]=fw[1][8-i][j]; // 朝下
    for(int i=1;i<=7;i++) for(int j=1;j<=7;j++) fw[3][i][j]=fw[1][j][i]; // 朝左
    for(int i=1;i<=7;i++) for(int j=1;j<=7;j++) fw[4][i][j]=fw[1][8-j][i]; // 朝右
    int n;
    cin>>n;
    for(int t=0;t<n;t++)
    {
        int x,y,f;
        cin>>x>>y>>f;
        for(int i=x-3;i<=x+3;i++) // 因为刚才已经旋转好了所以直接涂色即可
            for(int j=y-3;j<=y+3;j++)
                if(!(i<1||i>10||j<1||j>10)) // 如果没涂到战场范围之外的话
                    a[i][j]|=fw[f][i+4-x][j+4-y]; // 就涂色即可(若之前是0我们要把它变成1,若之前是1则还是1,用或即可)
    }
    for(int i=1;i<=10;i++)
    {
        for(int j=1;j<=10;j++)
            cout<<a[i][j];
        cout<<endl;
    }
    return 0;
}
\end{lstlisting}

\makesolution
\section*{做法}
\emph{本题改编自[CSP-S2019 江西] 和积和。}

本题有多种解法,这里提供一种效率不高但比较好想的前缀和做法。

设区间限制为 $[l,r]$ 时,玩家玩游戏的会获得的分数为 $S(l,r)$,即:

$$S(l,r) = \sum_{i=l}^r \sum_{j=l}^r a_i b_j$$

设我们需要求解的答案——所有可能的区间约束下每次你能获得的分数的总和——为 $ans$,有:

$$ans = \sum_{l=1}^n \sum_{r=l}^n S(l,r)$$

让我们来尝试着化简一下这个式子。首先,根据分配律,有:

$$S(l,r) = \sum_{i=l}^ra_i \times \sum_{j=l}^rb_j$$

设数列 $\{a_n\}$ 的前缀和数列为 $\{pre^a_n\}$,数列 $\{b_n\}$ 的前缀和数列为 $\{pre^b_n\}$,有:

\begin{align*}
    S(l,r) & = \left(pre^a_r - pre^a_{l-1}\right) \times \left(pre^b_r - pre^b_{l-1}\right)      \\
           & = pre^a_rpre^b_r - pre^a_rpre^b_{l-1} - pre^a_{l-1}pre^b_r + pre^a_{l-1}pre^b_{l-1}
\end{align*}

设 $pre^{ab}_i = pre^a_ipre^b_i$,有:

$$S(l,r) = pre^{ab}_r + pre^{ab}_{l-1} - pre^a_rpre^b_{l-1} - pre^a_{l-1}pre^b_r$$

把 $\sum_{r=l}^n$ 拆进去,有:

\begin{align*}
    ans & = \sum_{l=1}^n \left(
    \sum_{r=l}^n pre^{ab}_r +
    (n-l+1)pre^{ab}_{l-1} -
    \sum_{r=l}^n pre^a_rpre^b_{l-1} -
    \sum_{r=l}^n pre^a_{l-1}pre^b_r\right
    )                           \\
        & = \sum_{l=1}^n \left(
    \sum_{r=l}^n pre^{ab}_r +
    (n-l+1)pre^{ab}_{l-1} -
    pre^b_{l-1}\sum_{r=l}^n pre^a_r -
    pre^a_{l-1}\sum_{r=l}^n pre^b_r\right
    )
\end{align*}

设数列 $\{pre^a_n\}$ 的前缀和数列为 $\{pre^{pre^a}_n\}$,数列 $\{pre^b_n\}$ 的前缀和数列为 $\{pre^{pre^b}_n\}$,数列 $\{pre^{ab}_n\}$ 的前缀和数列为 $\{pre^{pre^{ab}}_n\}$,上式即化为:

\small
$$
    \sum_{l=1}^n \left(
    \left(pre^{pre^{ab}}_n - pre^{pre^{ab}}_{l-1}\right) +
    (n-l+1)pre^{ab}_{l-1} -
    \left(pre^{pre^a}_n - pre^{pre^a}_{l-1}\right)pre^a_r -
    \left(pre^{pre^b}_n - pre^{pre^b}_{l-1}\right)pre^b_r
    \right)
$$
\normalsize

数列 $a\{n\}$ 和数列 $b\{n\}$是已知的,我们可以 $O(n)$ 预处理出 $\{pre^a_n\}$ 与 $\{pre^b_n\}$,然后我们又可以 $O(n)$ 预处理出 $\{pre^{pre^a}_n\}$、$\{pre^{pre^b}_n\}$、$\{pre^{ab}_n\}$。这样上式后半部分的一大坨就可以 $O(1)$ 得出结果了,总时间复杂度 $O(n)$。

在实现的过程中需要注意本来原式只有加法,无论怎样式子中都不会出现负数。但转化为前缀和后式子中出现了减法,在值取模后相减有可能出现负数,需要把负数转回整数再继续运算。

\section*{标程}
\begin{lstlisting}
#include <iostream>
using namespace std;
const int MAXN = 5e5+5;
const int MOD = 1e9+7;
typedef long long ll;
ll a[MAXN],b[MAXN];
ll prea[MAXN],preb[MAXN],preab[MAXN];
ll preprea[MAXN],prepreb[MAXN],prepreab[MAXN];
int n;
inline void calcPre(ll a[], ll pre[])   // 计算a数组的前缀和存于pre数组中
{
    for(int i=1;i<=n;i++)
        pre[i]=(a[i]+pre[i-1]) % MOD;
}
ll solve()
{
    ll sum = 0;
    for(int l=1;l<=n;l++)
    {
        sum = (sum + prepreab[n]-prepreab[l-1]+MOD) % MOD;  // 第一项
        sum = (sum + (n-l+1)*preab[l-1]%MOD) % MOD;         // 第二项
        sum -= (preprea[n]-preprea[l-1])*preb[l-1]%MOD;     // 第三项
        sum = (sum + MOD) % MOD;    // 有可能出现负数,转为正数
        sum -= (prepreb[n]-prepreb[l-1])*prea[l-1]%MOD;     // 第四项
        sum = (sum + MOD) % MOD;    // 有可能出现负数,转为正数
    }
    return sum;
}
int main()
{
    cin>>n;
    for(int i=1;i<=n;i++) cin>>a[i];
    for(int i=1;i<=n;i++) cin>>b[i];
    calcPre(a, prea);
    calcPre(b, preb);
    for(int i=1;i<=n;i++) preab[i]=prea[i]*preb[i]%MOD;
    calcPre(prea, preprea);
    calcPre(preb, prepreb);
    calcPre(preab, prepreab);
    cout<<solve()<<endl;
    return 0;
}
\end{lstlisting}

\makesolution
\section*{做法}
签到题,按照题目说明中给出的海伦公式进行模拟即可。

\section*{标程}
\begin{lstlisting}
#include <cstdio>
#include <cmath>
int main()
{
    int t;
    scanf("%d", &t);
    while(t--)
    {
        int a,b,c;
        scanf("%d%d%d",&a,&b,&c);
        double p = (a+b+c)/2.0;
        double s = sqrt(p*(p-a)*(p-b)*(p-c));
        printf("%.2f\n", s);
    }
    return 0;
}
\end{lstlisting}

\makesolution
\section*{做法}
求各子树大小是树上的经典基操,一遍dfs即可解决。

\section*{标程}
\begin{lstlisting}
#include <iostream>
using namespace std;
const int maxn = 1e4+5;
struct E { int to,next; } Edge[maxn<<1];    // 链式前向星,因为是树所以开2倍maxn大小即可
int tot,Head[maxn];
inline void AddEdge(int u,int v) { Edge[tot]={v,Head[u]}; Head[u]=tot++; }
int siz[maxn];
void dfs(int u,int f)
{
    siz[u]=1;   // 每个结点的大小等于自己的所有子树的大小之和加1(因为还包括结点自己)
    for(int i=Head[u];~i;i=Edge[i].next)
    {
        int v = Edge[i].to;
        if(v==f) continue;  // 防止走回父亲
        dfs(v,u);   // 计算v子树大小
        siz[u]+=siz[v];     // 在u子树大小中加上v子树大小
    }
}
#include <cstring>
int main()
{
    memset(Head,-1,sizeof(Head));   // 链式前向星-1写法,将Head数组初始化为-1
    int n; cin>>n;
    for(int i=1,u,v;i<n;i++)        // 读入数据,注意正反加两条边
        cin>>u>>v, AddEdge(u,v), AddEdge(v,u);
    for(int rt=1;rt<=n;rt++)
    {
        dfs(rt, 0);
        for(int i=1;i<=n;i++)
            cout<<siz[i]<<' ';
        cout<<endl;
    }
    return 0;
}
\end{lstlisting}

\makesolution
\section*{做法}
\section*{标程}

\makesolution
\section*{做法}
首先,若落下的这着棋其周围没有己方棋子,那么它能产生的影响只有使得己方棋块数量增加了 $1$。

否则,我们就需要判断这着棋是不是一着“粘”,若这着棋是“粘”,它“粘”住了几块棋(也就是这着棋使得棋块的数量减少了多少)。

判断一着棋是不是一着“粘”即判断这着棋是否将不同的块连接了起来,也就是判断这着棋周围的己方棋子是否全部属于同一个块,显然我们可以使用并查集的查询操作来解决这个问题。而计算这着棋使得棋块的数量减少了多少更加容易,若其导致了一次并查集之间的合并,那么棋块的数量显然就会减一。

注意黑白双方是独立的,所以需要开两个并查集。每着棋的信息是由横纵坐标 $x,y$ 两个信息决定的,我们可以将其转化为 $(x-1)*n + y$ 来轻松搞定并查集的实现。

\section*{标程}
\begin{lstlisting}
#include <iostream>
#include <cstring>
using namespace std;
const int maxn = 505;
int fa[2][maxn*maxn];       // 两个并查集,一维下标为0时是白子的并查集,1时是黑子的并查集。
int find(int k, int x) { return x==fa[k][x]?x:x=find(k, fa[k][x]); }
inline void merge(int k,int x,int y) { fa[k][find(k,x)]=find(k,y); }
int n,m,zhan[2], kuai[2], board[maxn][maxn];
inline int hs(int x,int y) { return (x-1)*n+y; }
\end{lstlisting}
\clearpage
\begin{lstlisting}
int main()
{
    memset(board, -1, sizeof(board));
    cin>>n>>m;
    for(int i=1;i<=n*n;i++) fa[0][i]=fa[1][i]=i;    // 初始化并查集
    for(int i=1;i<=m;i++)
    {
        int x,y;
        cin>>x>>y;
        int k = i%2;
        board[x][y]=k;  // 存下当前位置棋子黑白
        ++kuai[k];      // 如果是孤立棋子的话块数就会+1
        int cnt = 0;    // 每与周围同色棋子合并一下cnt就+1,当cnt为2时代表这一着棋是粘
        if(board[x-1][y]==board[x][y]&&find(k, hs(x,y))!=find(k, hs(x-1,y)))    // 上
        {
            merge(k, hs(x,y), hs(x-1,y));
            --kuai[k];  // 每合并一次,块数就会减一
            ++cnt;
        }
        if(board[x+1][y]==board[x][y]&&find(k, hs(x,y))!=find(k, hs(x+1,y)))    // 下
        {
            merge(k, hs(x,y), hs(x+1,y));
            --kuai[k];
            if(++cnt==2) ++zhan[k];
        }
        if(board[x][y-1]==board[x][y]&&find(k, hs(x,y))!=find(k, hs(x,y-1)))    // 左
        {
            merge(k, hs(x,y), hs(x,y-1));
            --kuai[k];
            if(++cnt==2) ++zhan[k];
        }
        if(board[x][y+1]==board[x][y]&&find(k, hs(x,y))!=find(k, hs(x,y+1)))    // 右
        {
            merge(k, hs(x,y), hs(x,y+1));
            --kuai[k];
            if(++cnt==2) ++zhan[k];
        }
    }
    cout<<zhan[1]<<' '<<zhan[0]<<endl<<kuai[1]<<' '<<kuai[0]<<endl;
    return 0;
}
\end{lstlisting}

\end{document}