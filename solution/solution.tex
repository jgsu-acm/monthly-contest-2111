\documentclass{../cpct/ctsol}

\title{ACM 算法与微应用实验室 2021 年 11 月月赛题解}
\date{2021 年 12 月 1 日}

\begin{document}
\maketitle
\addsolution{克隆干员}{AgOH}{模拟}
\addsolution{中转站}{AgOH \& Tifa}{前缀和}
\addsolution{三斜求积术}{AgOH}{模拟}
\addsolution{子树大小}{AgOH}{dfs}
\addsolution{雷立方体阵列}{Tifa}{线段树}
\addsolution{Go}{AgOH}{并查集}

\section*{题目概览}

\solutiontab

\section*{鸣谢}

感谢\href{https://github.com/Tiphereth-A}{\@Tifa}大佬参与本次比赛的出题工作。

\makesolution
\section*{做法}

模拟过程。比较简便的做法如下:

首先我们先将干员的四种不同朝向时的状况分别保存于数组中(也就是先把干员旋转好),然后开出一个初始值为0的数组用来表示战场,最后每输入一个点后就向这个战场数组中涂色(填1)即可。

\section*{标程}

\std{A}

\makesolution
\section*{做法}

\emph{本题改编自[CSP-S2019 江西] 和积和。}

本题有多种解法,这里提供一种效率不高但比较好想的前缀和做法。

设区间限制为 $[l,r]$ 时,玩家玩游戏的会获得的分数为 $S(l,r)$,即:

$$S(l,r) = \sum_{i=l}^r \sum_{j=l}^r a_i b_j$$

设我们需要求解的答案——所有可能的区间约束下每次你能获得的分数的总和——为 $ans$,有:

$$ans = \sum_{l=1}^n \sum_{r=l}^n S(l,r)$$

让我们来尝试着化简一下这个式子。首先,根据分配律,有:

$$S(l,r) = \sum_{i=l}^ra_i \times \sum_{j=l}^rb_j$$

设数列 $\{a_n\}$ 的前缀和数列为 $\{pre^a_n\}$,数列 $\{b_n\}$ 的前缀和数列为 $\{pre^b_n\}$,有:

\begin{align*}
    S(l,r) & = \left( pre^a_r - pre^a_{l-1} \right) \times \left( pre^b_r - pre^b_{l-1} \right)      \\
           & = pre^a_rpre^b_r - pre^a_rpre^b_{l-1} - pre^a_{l-1}pre^b_r + pre^a_{l-1}pre^b_{l-1}
\end{align*}

设 $pre^{ab}_i = pre^a_ipre^b_i$,有:

$$S(l,r) = pre^{ab}_r + pre^{ab}_{l-1} - pre^a_rpre^b_{l-1} - pre^a_{l-1}pre^b_r$$

把 $\sum_{r=l}^n$ 拆进去,有:

\begin{align*}
    ans & = \sum_{l=1}^n \left( \sum_{r=l}^n pre^{ab}_r + (n-l+1)pre^{ab}_{l-1} - \sum_{r=l}^n pre^a_rpre^b_{l-1} - \sum_{r=l}^n pre^a_{l-1}pre^b_r \right) \\
        & = \sum_{l=1}^n \left( \sum_{r=l}^n pre^{ab}_r + (n-l+1)pre^{ab}_{l-1} - pre^b_{l-1}\sum_{r=l}^n pre^a_r - pre^a_{l-1}\sum_{r=l}^n pre^b_r \right)
\end{align*}

设数列 $\{pre^a_n\}$ 的前缀和数列为 $\{pre^{pre^a}_n\}$,数列 $\{pre^b_n\}$ 的前缀和数列为 $\{pre^{pre^b}_n\}$,数列 $\{pre^{ab}_n\}$ 的前缀和数列为 $\{pre^{pre^{ab}}_n\}$,上式即化为:

{
    \small
    $$
        \sum_{l=1}^n \left(
        \left(pre^{pre^{ab}}_n - pre^{pre^{ab}}_{l-1}\right) +
        (n-l+1)pre^{ab}_{l-1} -
        \left(pre^{pre^a}_n - pre^{pre^a}_{l-1}\right)pre^a_r -
        \left(pre^{pre^b}_n - pre^{pre^b}_{l-1}\right)pre^b_r
        \right)
    $$
}

数列 $\{a_n\}$ 和数列 $\{b_n\}$是已知的,我们可以 $O(n)$ 预处理出 $\{pre^a_n\}$ 与 $\{pre^b_n\}$,然后我们又可以 $O(n)$ 预处理出 $\{pre^{pre^a}_n\}$、$\{pre^{pre^b}_n\}$、$\{pre^{ab}_n\}$。这样上式后半部分的一大坨就可以 $O(1)$ 得出结果了,总时间复杂度 $O(n)$。

在实现的过程中需要注意本来原式只有加法,无论怎样式子中都不会出现负数。但转化为前缀和后式子中出现了减法,在值取模后相减有可能出现负数,需要把负数转回正数再继续运算。

\section*{标程}

\std{B}

\makesolution
\section*{做法}

签到题,按照题目说明中给出的海伦公式进行模拟即可。

\section*{标程}

\std{C}

\makesolution
\section*{做法}

求各子树大小是树上的经典基操,一遍 dfs 即可解决。

\section*{标程}

\std{D}

\makesolution
\section*{做法}

很容易能看出来本题是对数组 $a_1,a_2,\cdots,a_n$ 维护单点修改和区间查最大值(带单点修改的 RMQ 问题)。

做法有很多,此处给出线段树做法。

\section*{标程}

\std{E}

\makesolution
\section*{做法}

首先,若落下的这着棋其周围没有己方棋子,那么它能产生的影响只有使得己方棋块数量增加了 $1$。

否则,我们就需要判断这着棋是不是一着“粘”,若这着棋是“粘”,它“粘”住了几块棋(也就是这着棋使得棋块的数量减少了多少)。

判断一着棋是不是一着“粘”即判断这着棋是否将不同的块连接了起来,也就是判断这着棋周围的己方棋子是否全部属于同一个块,显然我们可以使用并查集的查询操作来解决这个问题。而计算这着棋使得棋块的数量减少了多少更加容易,若其导致了一次并查集之间的合并,那么棋块的数量显然就会减一。

每着棋的信息是由横纵坐标 $x,y$ 两个信息决定的,我们可以将其转化为 $(x-1)*n + y$ 来轻松搞定并查集的实现。

\section*{标程}

\std{F}

\end{document}
