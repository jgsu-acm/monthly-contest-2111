\documentclass{ctsol}

\title{ACM算法与微应用实验室2021年11月月赛题解}
\date{2021年12月1日}

\begin{document}
\maketitle
\addsolution{pro1}{AgOH}{模拟}
\addsolution{pro2}{AgOH}{模拟}
\addsolution{三斜求积术}{AgOH}{模拟}
\addsolution{子树大小}{AgOH}{dfs}
\addsolution{pro5}{AgOH}{模拟}
\addsolution{pro6}{AgOH}{模拟}

\section*{题目概览}
\solutiontab

\makesolution
\section*{做法}
\section*{标程}

\makesolution
\section*{做法}
\section*{标程}

\makesolution
\section*{做法}
签到题,按照题目说明中给出的海伦公式进行模拟即可。

\section*{标程}
\begin{lstlisting}
#include <cstdio>
#include <cmath>
int main()
{
    int t;
    scanf("%d", &t);
    while(t--)
    {
        int a,b,c;
        scanf("%d%d%d",&a,&b,&c);
        double p = (a+b+c)/2.0;
        double s = sqrt(p*(p-a)*(p-b)*(p-c));
        printf("%.2f\n", s);
    }
    return 0;
}
\end{lstlisting}

\makesolution
\section*{做法}
求各子树大小是树上的经典基操,一遍dfs即可解决。

\section*{标程}
\begin{lstlisting}
#include <iostream>
using namespace std;
const int maxn = 1e4+5;
struct E { int to,next; } Edge[maxn<<1];    // 链式前向星,因为是树所以开2倍maxn大小即可
int tot,Head[maxn];
inline void AddEdge(int u,int v) { Edge[tot]={v,Head[u]}; Head[u]=tot++; }
int siz[maxn];
void dfs(int u,int f)
{
    siz[u]=1;   // 每个结点的大小等于自己的所有子树的大小之和加1(因为还包括结点自己)
    for(int i=Head[u];~i;i=Edge[i].next)
    {
        int v = Edge[i].to;
        if(v==f) continue;  // 防止走回父亲
        dfs(v,u);   // 计算v子树大小
        siz[u]+=siz[v];     // 在u子树大小中加上v子树大小
    }
}
#include <cstring>
int main()
{
    memset(Head,-1,sizeof(Head));   // 链式前向星-1写法,将Head数组初始化为-1
    int n; cin>>n;
    for(int i=1,u,v;i<n;i++)        // 读入数据,注意正反加两条边
        cin>>u>>v, AddEdge(u,v), AddEdge(v,u);
    for(int rt=1;rt<=n;rt++)
    {
        dfs(rt, 0);
        for(int i=1;i<=n;i++)
            cout<<siz[i]<<' ';
        cout<<endl;
    }
    return 0;
}
\end{lstlisting}

\makesolution
\section*{做法}
\section*{标程}

\makesolution
\section*{做法}
\section*{标程}

\end{document}